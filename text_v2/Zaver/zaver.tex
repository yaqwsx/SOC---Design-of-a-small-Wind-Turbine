\chapter{Zhodnocení práce}
Cílem této práce bylo seznámit čtenáře s aerodynamikou malých větrných turbín a ukázat její použití v praxi – jak při návrhu, tak i samotné stavbě. Tyto cíle se podařilo splnit.

V této práci jsem na základě Glauertovy teorie navrhl větrnou malou větrnou turbínu. Tuto teorii jsem doplnil o poznatky získané z předchozí stavby prvního prototypu a zkusil jsem na základě aerodynamických simulací optimalizovat zakončení listu větrné turbíny.

\chapter{Budoucnost projektu}

Jak je patrné z celé práce, práce na projektu malé větrné elektrárny není hotová a vyžaduje ještě spoustu času.

V budoucnu by měla být výše navržená turbína vyrobena. Technologie výroby zatím není známa. Pokud to dovolí prostředky, měly by listy být vyrobené z laminátu.

Tato turbína bude připojena na pomaloběžný generátor, jehož vývoj je téměř u konce. Momentálně se nachází ve fázi testování. Energie vyrobená touto elektrárnou by ke své povaze (nestálé frekvenci a napětí) měla být použita k dotápění domu či ohřevu vody.

K elektrárně je také nutno dodělat komplexní ochranný systém před vichřicí a dalšími vlivy. Jelikož výkon turbíny neroste s otáčkami (otáčky rostou s rychlostí větru lineárně, výkon s třetí mocninou) je nutné přidat elektronické spínání zátěže generátoru, aby byla turbína efektivně využita. Elektrárna by také měla být doplněna o čidla a vybavena telemetrií s ukládáním dat a webovým rozhraním. Tento systém telemetrie mi již částečně funguje na pokusném anemometru. Je založen na routeru Asus WL-500GP. Avšak mám v plánu tento systém přestavět na platformu ARM, konkrétně na mikroprocesory STM32 kvůli jejich minimální spotřebě, velikosti a ceně v~porovnání s routerem. Mikroprocesor se navíc lépe zabudovává do embedded systému. Elektrárna by také měla jít z webového rozhraní ovládat – např. ji odstavit z provozu.


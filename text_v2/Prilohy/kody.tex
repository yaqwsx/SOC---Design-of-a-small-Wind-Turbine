\chapter{Zdrojové kódy programu pro výpočet}
V následující příloze jsou uvedeny zdrojové kódy k programu pro výpočet dle Glauertovy teorie. Program je napsán pro C++, kód je komentován, tudíž by měl být snadno srozumitelný.

Program nepřebírá žádný uživatelský vstup, data jsou zadávána přímo do zdrojového kódu, jelikož program je jednoúčelový. Výstupem je tabulka tabulátorem oddělených dat v souboru out.txt.

V programu je záměrně pro názornost použit desítkový základ pro inkrement, ačkoliv plně nevyužívá přesnosti čísla s~plovoucí desetinou čárkou (vzniká zde zaokrouhlovací chyba). Pro zvýšení přesnosti je nutné použít dvojkový základ (tedy číslo ve formátu $2^n$, nikoliv $10^n$). Avšak i takto je výpočet prováděn s větší přesností, než je v praxi potřeba.

Program je krátký, je rozdělen do 3 souborů: main.cpp, functions.h a functions.cpp. V obsahu souborů by neměl být problém se zorientovat.

\section{main.cpp}
\begin{mylisting}
\begin{verbatim}
#define _USE_MATH_DEFINES
#include <iostream>
#include <fstream>
#include "functions.h"
#include <iomanip>

using namespace std;

const char t = '\t';//Definování tabulátoru pro zkrácení kódu

const unsigned int presicion = 10;//Definuje počet platných číslic pro výstup
const unsigned int z = 3;//Počet lopatek turbíny
const double thickness = 0.1;//Tloušťka profilu v poměru ku délce
const double cy = 1.303;//Součinitel vztlaku
const double cx = 0.017;//Součinitel odporu
const double E = tan(cx/cy);//Jemnost profilu
const double L = 5;//Rychloběžnost
const double R = 1.25;//Poloměr turbíny
const size_t fractions = 200;//Počet segmentů, na které je list pro výpočet rozdělen
const double rIncrement = R/fractions;//Definuje inkrement poloměru při výpočtu
double const A = 5.5/180*M_PI;//Ideální úhel náběhu profilu v radiánech
const double frontAspect = 1.0/3.0;//Udává vzdálenost osy, okolo které se otáčí profil; od počátku

int main()
{
       ofstream o("out.txt");//Výstupní soubor
       double r = rIncrement*3;//První 3 segmenty u středu přeskakuji
               // - nejsou použity a jich výpočet trvá příliš dlouho
       size_t counter = 0;//Pro účely debuggování
       o << "Polomer\th\tk\tB\tCp\tb\tz1\ty1\tz2\ty2\tx" << endl;//Nadpis tabulky
       for(r; r <= R; r += rIncrement)
       {
               counter++;
               cout << "Pocitam " << counter << t;
               data Data = CountCoefficients(E, r/R*L);
               cout << "Hotovo" << endl;
               double b = 8*M_PI*r*cos(E)*(Data.h-1)*sin(Data.B)*cos(Data.B)/
                  (sin(Data.B-E)*(Data.h+1))/(z*cy);//Výpočet délky tětivy
               double z1, z2, y1, y2;
               //Výpočet souřadnic křivek náběžné a odtokové hrany
               //násobení 1000 je zde k převedení rozměrů na milimetry pro CAD program
               y1 = (b*frontAspect*sin(Data.B-A)+cos(Data.B-A)*b*thickness/2)*1000;
               y2 = (-b*(1-frontAspect)*sin(Data.B-A)+cos(Data.B-A)*b*thickness/2)*1000;
               z1 = (b*frontAspect*cos(Data.B-A)-sin(Data.B-A)*b*thickness/2)*1000;
               z2 = (-b*(1-frontAspect)*cos(Data.B-A)-sin(Data.B-A)*b*thickness/2)*1000;
               //Nastavení přesnosti
               o << setprecision(presicion);
               //Výpis dat
               o << r << t << Data.h << t << Data.k << t << Data.B << t << Data.Cp << t << b << t
                       << z1 << t << y1 << t << z2 << t << y2 << t << r*1000 << endl;
       }
       return 0;
}

\end{verbatim}
\end{mylisting}

\section{functions.h}
\begin{mylisting}
\begin{verbatim}
#pragma once

#define _USE_MATH_DEFINES
#include <cmath>

const double incrementLimit = pow(10.0, -16);//Definuje přesnost celého výpočtu
const double differentionLimit = pow(10.0, -9);//Definuje přesnost výpočtu k a beta
const double incrementStep = 10;//Definuje základní krok inkrementu př změně směru
const double incrementDefault = pow(10.0, -6);//Počáteční velikost inkrementu
extern double increment;

struct data
{
       double B, k, h, Cp;
       /*B je beta, k a h jsou koeficienty a Cp je součinitel výkonu*/
};

data CountCoefficients(double E, double l);//Funkce spočítá data pro zadané epsilon a rychloběžnost na poloměru r

\end{verbatim}
\end{mylisting}

\section{functions.cpp}
\begin{mylisting}
\begin{verbatim}
#include "functions.h"

double increment;//Definice proměnné inkremetu

data CountCoefficients(double E, double l)
{
       increment = incrementDefault;//Natavení inkrementu
       data ret;//Data k navrácení
       //Počáteční nastavení
       ret.k = 1.0/3.0;
       ret.h = 1 +  pow(10.0, -5);
       ret.Cp = 0;

       //Nastavení bety na první hodntou
       ret.B = atan((1+ret.k)/(1+ret.h)/l);
       bool last = true;//definuje, zda-li minulá změna inkrementu byla kladná (true) nebo záporná (false)
       size_t iterationCount = 0;//Pro účely debuggování

       while(increment >= incrementLimit)//Dokud inkrement nedosáhne daného řádu...
       {
               //Dopočítání k
               double differention;
               double kn = ret.k;
               size_t iterationCount2 = 0;
               do
               {
                       ret.k = kn;
                       kn = 1 - (ret.h-1)*l/tan(ret.B-E);
                       ret.B = atan((1+kn)/(1+ret.h)/l);
                       iterationCount2++;
                       differention = atan((1+kn)/(1+ret.h)/l)-atan((1+ret.k)/(1+ret.h)/l);
               }while(differention > differentionLimit);

               //Určení součinitele výkonu
               double c = l*l*(1+kn)*(ret.h-1);
               if(c > ret.Cp)
               {
                       //Součinitel je větší než předchozí
                       if(last != true)
                               increment /= incrementStep;//Změna směru, snížení řádu inkrementu
                       last = true;
                       ret.h += increment;
               }
               else
               {
                       //Součinitel je menší než předchozí
                       if(last != false)
                               increment /= incrementStep;//Změna směru, snížení řádu inkrementu
                       ret.h -= increment;
               }

               ret.Cp = c;
               ret.k=kn;

               iterationCount++;
       }
       return ret;

\end{verbatim}
\end{mylisting}